% LaTeX template for Finance summaries
% Stu Alden
%----------------------------------------------------------------------------------------
% Bonds and Stocks
\documentclass[12pt]{article}

\usepackage[letterpaper,
            left=1in,
            right=1in,
            top=1in,
            bottom=1in,
            footskip=.25in]{geometry}

\usepackage{titling}
\setlength{\droptitle}{-1in}  % Don't need all that space!
\title{\normalfont\ 4-A Bonds and Stocks} % The article title
\author{} % Don't need author
\date{}  % Suppress date
\pagenumbering{gobble}  % No page numbers required

\usepackage{amsmath}
\usepackage{actuarialsymbol}   % for the angle sign on the annuity

\usepackage{tikz}
\usetikzlibrary{decorations.pathmorphing}

\begin{document}

\maketitle % Print the title/author/date block

\vspace{-1.1in}

\begin{flushleft}
    We can analyze {stocks} and {bonds} using all of the compound interest techniques we've
    learned so far:
\end{flushleft}
\vspace{.05in}

\begin{itemize}
    \item A bond (purchased at issue, or later on the seconday market) can be viewed as a combination of
    \begin{itemize}
        \item an annuity of coupon payments, plus
        \item a lump sum payment of the face amount at maturity
    \end{itemize}
    \item A stock can be viewed as
    \begin{itemize}
        \item an annuity of dividends (if applicable), perhaps varying over time, plus
        \item the proceeds from the sale of the stock at some future date.
    \end{itemize}
\end{itemize}



% \begin{center}
%     \begin{tikzpicture}
%         %draw horizontal line with squiggle in middle
%         \draw (0,0) -- (5,0);
%         \draw[decorate,decoration={snake,pre length=5mm, post length=5mm}] (5,0) -- (7,0);
%         \draw (7,0) -- (10,0);

%         %draw vertical tick-marks
%         \foreach \x in {0,2,4,8,10}
%         \draw (\x cm,3pt) -- (\x cm,-3pt);

%         %Label "nodes"
%         \draw (0,0) node[below=3pt] {$ 0 $} node[above=3pt] {$  $};
%         \draw (0,0) node[below=18pt] {$ A $} node[above=3pt] {$  $};

%         \draw (2,0) node[below=3pt] {$ 1 $} node[above=3pt] {$ R $};
%         \draw (4,0) node[below=3pt] {$ 2 $} node[above=3pt] {$ R $};
%         \draw (6,0) node[below=3pt] {$  $} node[above=3pt] {$  $};
%         \draw (8,0) node[below=3pt] {$ n - 1 $} node[above=3pt] {$ R $};
%         \draw (10,0) node[below=3pt] {$ n $} node[above=3pt] {$ R $};
%         \draw (10,0) node[below=18pt] {$ S $} node[above=3pt] {$ R $};
%     \end{tikzpicture}
% \end{center}
% %\vspace{.1in}

\begin{flushleft}
    Bond cashflows are generally far more predictable than stock cashflows, but the principles
    for analysis are the same.  In most cases we either have a desired a yield rate (i) and we are trying
    to solve for A (the price, namely the present value of the cashflows at the time of purchase) or we know A (The
    purchase price) and we want to estimate the yield.  Either way, we can use compound interest and
    the annuity formula.  If
\end{flushleft}
\vspace{.25in}

\begin{description}
    \item \textbf{R} - Amount of each payment (coupon or dividend)
    \item\textbf{n} - Number of periods until maturity or sale of stock
    \item\textbf{i} - Compound interest rate
    \item\textbf{A} - Present Value (purchase price) of the cashflows
    \item\textbf{M} - Maturity value of bond / sale price of stock
\end{description}
\vspace{.25in}

\begin{flushleft}
    Then
\end{flushleft}



% \begin{flushleft}
%     S denotes the ending, or \textbf{accumulated value} of the stream of annuity payments. To value instead
%     the payments at the beginning (denoted \textbf{present value}), we just discount S by n periods, as follows:
% \end{flushleft}
% \vspace{-.1in}

\begin{align*}
    % A & = \frac{S}{(1 + i)^n}      \\\\
    %   & = Sv^n                     \\\\
    A & = R \cdot \frac{1 - v^n}{i} + M \cdot v^n
\end{align*}

%    & = R \cdot \frac{1 - v^n}{i}

% \begin{align*}
%     S & = R \cdot \frac{(1 + i)^n-1}{i} \\
% \end{align*}
% \vspace{-.2in}
\begin{flushleft}
    where {\large $ v = \frac{1}{(1+i)} $ }.
\end{flushleft}
% \vspace{-.1in}
\vspace{.1in}

\begin{flushleft}
    \textbf{Note:} \\
\end{flushleft}

\begin{flushleft}
    The price calculation (A) is generally straightforward.  The yield calculation, on the other hand,
    is generally not feasible with direct ("analytic") methods, but is easy to estimate to any
    required precision with the aid of computers.

    % When actuaries want to specify the interest rate and the term of the annuity in the notation, they use the symbol
    % {\Large $ \ax{ \angln i} $} for present value and {\Large $ \sx{ \angln i} $} for accumulated or future value, where
    % $ n $ is the number of periods and $ i $ is the rate per period.
\end{flushleft}

\end{document}