% Financial Mathematics
% Exercise Template
%----------------------------------------------
\documentclass[12pt]{exam}

\usepackage{amsmath}

\pagestyle{head}
\rhead{Name: \rule[-0.1cm]{5cm}{0.01cm}}
\lhead{Financial Literacy - Building Your Future\\Special Re-take Test}

\begin{document}

\begin{flushleft}
    One hour, 8 problems, 10 points each, 80 points total. \\
    Open notes and calculator
\end{flushleft}

\begin{questions}
    \question Your CVS receipt for some purchases you made in another state looks like this:
    \begin{itemize}
        \item Prescription A......\$39.00
        \item Prescription B......\$72.00
        \item Toothpaste...........\$ 4.95    tx
        \item Toothbrush...........\$ 6.95    tx
        \item Comb....................\$ 2.57    tx
        \item Distilled Water.....\$ 4.47    tx
    \end{itemize}

    A 'tx' next to an item means that it was subject to sales tax.
    If your total bill was \$131.36, what was the sales tax rate (answer using x.xx\% format)?
    \vspace{1in}

    \question Your taxable income, before any deductions, is \$32,000.  You qualify for EITHER a
    \$1,100 tax deduction OR a \$250 tax credit, whichever is more favorable to you.  Assuming your
    tax rate is 21\%, which option (deduction or credit) would  you choose, and what total tax
    amount will you pay?
    \vspace{1in}

    \question Compare and contrast bonds and stocks.  List as many characteristics and properties
    as you can think of for each type of security.
    \vspace{1in}

    \question Bob takes out a \$100,000, 15-year level-payment mortgage at 8\% interest compounded monthly on January 1.
    On February 1, he makes his first payment (principal and interest).  How much of that first payment
    is interest?
    \vspace{1in}

    \question Over n years, at 5\% interest, a dollar will grow to $ (1+.05 \cdot n) $ with simple interest
    and $ (1.05)^n $ under compound interest.
    \begin{itemize}
        \item Which method, simple or compound, pays more if your number of
            years is longer than one (say 2 or 5)?
        \item Which method, simple or compound, pays more if your period is only
            a fraction of a year (e.g., $ \frac{1}{2} $ year)?
    \end{itemize}
    \vspace{1.5in}

    \question List five different regular expenditures you make which you would classify as "needs." List five
    other regular expenditures you make which you would classify as "wants." What is the potential benefit to
    you if you were to reduce or eliminate the "want" expenditures?
    \vspace{1in}

    \question Inflation in Freedonia has been so high that prices have doubled in the past 6 years.
    What has been the average annual inflation over the 6-year period (answer as x.xx\%)?
    \vspace{1in}

    \question You are comparing a 15-year mortgage with a 30-year mortgage.  Assuming both
    both mortgages are for the same loan amount, both have level payments each month, and their
    interest rate is identical, identify each of these statements as (T)rue or (F)alse:
    \begin{itemize}
        \item T / F - The 30-year mortgage will have lower monthly payments.
        \item T / F - The 15-year mortgage will have you paying the principal off more slowly
        \item T / F - The 15-year mortgage will result in lower total interest payments
    \end{itemize}

\end{questions}

\end{document}