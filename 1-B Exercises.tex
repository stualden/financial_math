% Financial Mathematics
% Exercise Template
%----------------------------------------------
\documentclass[12pt]{exam}

\pagestyle{head}
\rhead{Name: \rule[-0.1cm]{5cm}{0.01cm}}
\lhead{Financial Literacy - Financial Mathematics\\Section 1-B:  Compound Interest}

\begin{document}

\begin{questions}

    \question If you put \$5 in an account earning 5\% per year compounded annually, and you leave it
    there for 5 years, how much money would you end up with?
    \vspace{2in}

    \question You need \$10,000 in 3 years in order to buy a used car. If you have an account
    that earns 6\% compounded monthly, how much money do you need to deposit today to
    produce the \$10,000 by the end of 3 years?
    \vspace{2in}

    \question 5\% nominal interest compounded quarterly is equivalent to X\% nominal interest
    compounded monthly. Find X (to 3 decimal places).
    \vspace{2in}

    \question You have an account that earns 10\% nominal annual interest, but you don’t know how
    often interest is compounded. If \$100 in this account grows to \$148.94 after 4 years, what
    was the compounding period?
    \vspace{1.5in}

    \question If the population of a city grows by 12\% per year, how many years (to one decimal place) will it take for
    the population to double in size?
    \vspace{1.5in}

    \question What is the effective annual rate of interest if you have a 10\% nominal annual
    rate compounded
    \begin{itemize}
        \item Semi-annually?
        \item Quarterly?
        \item Month?
        \item Daily?
    \end{itemize}
    \vspace{2in}

    \question At what nominal rate, compounded quarterly, would \$1,000 grow to \$1,500 in 5 years?
    \vspace{2in}

    \question If \$3,000 grows to \$4,000 over 10 years, what nominal annual rate, compounded
    monthly, did it earn?
    \vspace{2in}

\end{questions}

\end{document}
