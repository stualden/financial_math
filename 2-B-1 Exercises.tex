% Financial Mathematics
% Exercise Template
%----------------------------------------------
\documentclass[12pt]{exam}

\pagestyle{head}
\rhead{Name: \rule[-0.1cm]{5cm}{0.01cm}}
\lhead{Financial Literacy - Financial Mathematics\\Section 2-B-1:  Simple Annuities}

\begin{document}

\vspace{2in}  % doesn't seem to do anything...

\begin{questions}

    \question Find the accumulated value of a simple annuity of \$2,000 per year for 5 years
    if money is worth 9\% per year.
    \vspace{1.25in}

    \question It is estimated that a machine will need to be replaced 10 years from now at a cost of \$80,000.  How
    much must be set aside each year to provide that amount, if you can earn 8\% interest per year?
    \vspace{1.5in}

    \question Jeff buys a used car by paying \$3000 down and \$300 per month for 3 years.  What was the
    equivalent up-front cash price of the car, if the interest rate on the loan is 9\% compounded monthly?
    \vspace{2in}

    \question To settle a debt with interest at 12\% compounded semiannually, Leonard agrees to make 15
    payments of \$400 at the end of each half-year, and then a final payment, 6 months later, of \$292.39.
    What was the original amount of the debt?
    \vspace{2in}

    \question John is repaying a debt with payments of \$250 per month at the end of each month.  If he misses his
    payments for July, August, September and October, what payment will be required in November to put him back on
    schedule, if interest is 6\% compounted monthly?
    \vspace{2in}

    \question An annuity with payments at the end of each month pays \$200 for 2 years, then \$300 for the next year, then
    \$400 for the final 2 years.  Find the discounted value of these payments at 10\% compounded monthly.
    at the end of two years, and the balance at the end of three years.  What is the amount of that final payment?
    \vspace{1.5in}

    \question A company is trying to decide whether to buy or lease some equipment.
    \begin{itemize}
        \item They can buy it for \$40,000, and then incur maintenance costs of \$400 per month, payable at the end of each
              month.  At the end of 6 years, they expect to be able to sell the used equipment for \$5,000.
        \item Alternatively, they could lease the equipment \$1,200 per month, payable at the end of each month, for
              6 years, and the lessor would pay maintenance costs.  After 6 years, the lessor would take back the equipment
    \end{itemize}
    If the company can earn 18\% compounded monthly on capital, would it make more sense to buy or lease?
    \vspace{1.5in}

\end{questions}

\end{document}
