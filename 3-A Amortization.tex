% LaTeX template for Finance summaries
% Stu Alden
%----------------------------------------------------------------------------------------
% Amortization
\documentclass[12pt]{article}

\usepackage[letterpaper,
            left=1in,
            right=1in,
            top=1in,
            bottom=1in,
            footskip=.25in]{geometry}

\usepackage{titling}
\setlength{\droptitle}{-1in}  % Don't need all that space!
\title{\normalfont\ 3-A Amortization} % The article title
\author{} % Don't need author
\date{}  % Suppress date
\pagenumbering{gobble}  % No page numbers required

\usepackage{amsmath}
\usepackage{actuarialsymbol}   % for the angle sign on the annuity

\usepackage{tikz}
\usetikzlibrary{decorations.pathmorphing}

\begin{document}

\maketitle % Print the title/author/date block

\vspace{-1.1in}

\begin{flushleft}
    \textbf{Amortization} in its simplest form is the repayment of a debt in level installments, where
    each installment covers the full amount of interest that has accrued, plus a portion of the outstanding
    principal.  An \textbf{amortization schedule} is a table which can be created, showing each payment
    (the total, the interest portion, and the principal portion) and the outstanding principal immediately after
    the payment has been made.  If the outstanding principal is needed at any point during the term of
    the loan, it can be calculated using annuity formulas, either \textit{retrospectively} or
    \textit{prospectively}, as follows, where $ R $ is the periodic payment and $ n $ is the total
    number of payments over the life of the loan:
\end{flushleft}

\begin{align*}
    Outstanding Principal at time t & = R \cdot \ax{\angln-t i} \\
\end{align*}
\vspace{-.2in}

{\Large $ \ax{ \angln i} $} for present value and {\Large $ \sx{ \angln i} $}

\begin{flushleft}
    where {\large $ v = \frac{1}{(1+i)} $ }.
\end{flushleft}
% \vspace{-.1in}
\vspace{.1in}


\begin{description}
    \item \textbf{R} - Amount of each payment, assumed to be at the \underline{end} of each period
    \item\textbf{n} - Number of payments, and also (in the simple case) the number of periods
    \item\textbf{i} - Compound interest rate \underline{per period} expressed as a decimal {(\%/100)}
    \item\textbf{S} - Accumulated (or Future) Value of the annuity (value at time n)
    \item\textbf{A} - Present Value of the annuity (value at time 0)
\end{description}
\vspace{.1in}

\begin{center}
    \begin{tikzpicture}
        %draw horizontal line with squiggle in middle
        \draw (0,0) -- (5,0);
        \draw[decorate,decoration={snake,pre length=5mm, post length=5mm}] (5,0) -- (7,0);
        \draw (7,0) -- (10,0);

        %draw vertical tick-marks
        \foreach \x in {0,2,4,8,10}
        \draw (\x cm,3pt) -- (\x cm,-3pt);

        %Label "nodes"
        \draw (0,0) node[below=3pt] {$ 0 $} node[above=3pt] {$  $};
        \draw (0,0) node[below=18pt] {$ A $} node[above=3pt] {$  $};

        \draw (2,0) node[below=3pt] {$ 1 $} node[above=3pt] {$ R $};
        \draw (4,0) node[below=3pt] {$ 2 $} node[above=3pt] {$ R $};
        \draw (6,0) node[below=3pt] {$  $} node[above=3pt] {$  $};
        \draw (8,0) node[below=3pt] {$ n - 1 $} node[above=3pt] {$ R $};
        \draw (10,0) node[below=3pt] {$ n $} node[above=3pt] {$ R $};
        \draw (10,0) node[below=18pt] {$ S $} node[above=3pt] {$ R $};
    \end{tikzpicture}
\end{center}
%\vspace{.1in}

\begin{flushleft}
    Then we have
\end{flushleft}
\vspace{-.2in}

\begin{align*}
    S & = R \cdot \frac{(1 + i)^n-1}{i} \\
\end{align*}
\vspace{-.2in}

\begin{flushleft}
    S denotes the ending, or \textbf{accumulated value} of the stream of annuity payments. To value instead
    the payments at the beginning (denoted \textbf{present value}), we just discount S by n periods, as follows:
\end{flushleft}
\vspace{-.1in}

\begin{align*}
    A & = \frac{S}{(1 + i)^n}      \\\\
      & = Sv^n                     \\\\
      & = R \cdot \frac{1 - v^n}{i}
\end{align*}

%    & = R \cdot \frac{1 - v^n}{i}

\begin{flushleft}
    where {\large $ v = \frac{1}{(1+i)} $ }.
\end{flushleft}
% \vspace{-.1in}
\vspace{.1in}

\begin{flushleft}
    \textbf{Note:} \\
\end{flushleft}

\begin{flushleft}
    When actuaries want to specify the interest rate and the term of the annuity in the notation, they use the symbol
    {\Large $ \ax{ \angln i} $} for present value and {\Large $ \sx{ \angln i} $} for accumulated or future value, where
    $ n $ is the number of periods and $ i $ is the rate per period.
\end{flushleft}

\end{document}