% LaTeX template for Finance summaries
% Stu Alden
%----------------------------------------------------------------------------------------
% Simple Annuities
\documentclass[12pt]{article}

\usepackage[letterpaper,
            left=1in,
            right=1in,
            top=1in,
            bottom=1in,
            footskip=.25in]{geometry}

\usepackage{titling}
\setlength{\droptitle}{-1in}  % Don't need all that space!
\title{\normalfont\ 2-A Simple Annuities} % The article title
\author{} % Don't need author
\date{}  % Suppress date
\pagenumbering{gobble}  % No page numbers required

\usepackage{amsmath}
\usepackage{actuarialsymbol}   % for the angle sign on the annuity

\usepackage{tikz}
\usetikzlibrary{decorations.pathmorphing}

\begin{document}

\maketitle % Print the title/author/date block

\vspace{-1.1in}

\begin{flushleft}
    \textbf{Annuities} are simply a collection of multiple payments, generally (but not always)
    level amounts that are generally (but not always) equally spaced.  When amounts are level, and
    the rate period coincides with the payment period, we call them ``simple'' annuities.
\end{flushleft}

\begin{flushleft}
    The mathematics needed to value them is exactly the same as what we've already seen for compound interest,
    but when there are many payments, it is usually easier to use a calculator or computer.  There are some
    computational shortcuts, however, and that is the subject of this section.
\end{flushleft}
\vspace{.05in}

\begin{description}
    \item \textbf{R} - Amount of each payment, assumed to be at the \underline{end} of each period
    \item\textbf{n} - Number of payments, and also (in the simple case) the number of periods
    \item\textbf{i} - Compound interest rate \underline{per period} expressed as a decimal {(\%/100)}
    \item\textbf{S} - Accumulated (or Future) Value of the annuity (value at time n)
    \item\textbf{A} - Present Value of the annuity (value at time 0)
\end{description}
\vspace{.1in}

\begin{center}
    \begin{tikzpicture}
        %draw horizontal line with squiggle in middle
        \draw (0,0) -- (5,0);
        \draw[decorate,decoration={snake,pre length=5mm, post length=5mm}] (5,0) -- (7,0);
        \draw (7,0) -- (10,0);

        %draw vertical tick-marks
        \foreach \x in {0,2,4,8,10}
        \draw (\x cm,3pt) -- (\x cm,-3pt);

        %Label "nodes"
        \draw (0,0) node[below=3pt] {$ 0 $} node[above=3pt] {$  $};
        \draw (0,0) node[below=18pt] {$ A $} node[above=3pt] {$  $};

        \draw (2,0) node[below=3pt] {$ 1 $} node[above=3pt] {$ R $};
        \draw (4,0) node[below=3pt] {$ 2 $} node[above=3pt] {$ R $};
        \draw (6,0) node[below=3pt] {$  $} node[above=3pt] {$  $};
        \draw (8,0) node[below=3pt] {$ n - 1 $} node[above=3pt] {$ R $};
        \draw (10,0) node[below=3pt] {$ n $} node[above=3pt] {$ R $};
        \draw (10,0) node[below=18pt] {$ S $} node[above=3pt] {$ R $};
    \end{tikzpicture}
\end{center}
%\vspace{.1in}

\begin{flushleft}
    Then we have
\end{flushleft}
\vspace{-.2in}

\begin{align*}
    S & = R \cdot \frac{(1 + i)^n-1}{i} \\
\end{align*}
\vspace{-.2in}

\begin{flushleft}
    S denotes the ending, or \textbf{accumulated value} of the stream of annuity payments. To value instead
    the payments at the beginning (denoted \textbf{present value}), we just discount S by n periods, as follows:
\end{flushleft}
\vspace{-.1in}

\begin{align*}
    A & = \frac{S}{(1 + i)^n}      \\\\
      & = Sv^n                     \\\\
      & = R \cdot \frac{1 - v^n}{i}
\end{align*}

%    & = R \cdot \frac{1 - v^n}{i}

\begin{flushleft}
    where {\large $ v = \frac{1}{(1+i)} $ }.
\end{flushleft}
% \vspace{-.1in}
\vspace{.1in}

\begin{flushleft}
    \textbf{Note:} \\
\end{flushleft}

\begin{flushleft}
    When actuaries want to specify the interest rate and the term of the annuity in the notation, they use the symbol
    {\Large $ \ax{ \angln i} $} for present value and {\Large $ \sx{ \angln i} $} for accumulated or future value, where
    $ n $ is the number of periods and $ i $ is the rate per period.
\end{flushleft}

\end{document}