% LaTeX template for Finance summaries
% Stu Alden
% Amortization
%%%%%%%%%%%%%%%%%%%%%%%%%%%%%%%%%%%%%%%%%%%%%%%%%%%%%%%%%%%%%%%%%%%%%%%%
\documentclass[12pt]{article}

\usepackage[letterpaper,
            left=1in,
            right=1in,
            top=1in,
            bottom=1in,
            footskip=.25in]{geometry}

\usepackage{titling}
\setlength{\droptitle}{-1in}  % Don't need all that space!
\title{\normalfont\ 3-A Amortization} % The article title
\author{} % Don't need author
\date{}  % Suppress date
\pagenumbering{gobble}  % No page numbers required

\usepackage{amsmath}

\usepackage{actuarialsymbol}   % for the angle sign on the annuity!

\usepackage{tikz}
\usetikzlibrary{decorations.pathmorphing}

\begin{document} %%%%%%%%%%%%%%%%%%%%%%%%%%%%%%%%%%%%%%%%%%%%%%%%%%%%%%

\maketitle % Print the title/author/date block
\vspace{-1.1in}

\begin{flushleft}
    \textbf{Amortization} in its simplest form is the repayment of a debt with level payments ("installments"),
    where each installment covers the full amount of interest that has accrued, plus a portion of the outstanding
    principal.  An \textbf{amortization schedule} is a table with a line for each payment (showing the
    total payment, the interest portion, and the principal portion), and the outstanding principal immediately after
    that payment has been made.  To illustrate:
\end{flushleft}

\begin{flushleft}
    where {\large $ v = \frac{1}{(1+i)} $ }.
\end{flushleft}
% \vspace{-.1in}
\vspace{.1in}

\begin{flushleft}
    If the outstanding principal is needed at any point during the term of
    the loan, it can be calculated using the standard annuity formulas, either \textit{retrospectively} or
    \textit{prospectively}.  Here, $ R $ is the periodic payment, $ n $ is the total
    number of payments over the life of the loan, and $ t $ is the time at which the outstanding
    principal is measured:
\end{flushleft}

\begin{align*}
    Outstanding Principal at time t & = R \cdot \ax{\angln-t i} \\
\end{align*}
\vspace{-.2in}

{\Large $ \ax{ \angln i} $} for present value and {\Large $ \sx{ \angln i} $}

\begin{flushleft}
    where {\large $ v = \frac{1}{(1+i)} $ }.
\end{flushleft}
% \vspace{-.1in}
\vspace{.1in}


\begin{description}
    \item \textbf{R} - Amount of each payment, assumed to be at the \underline{end} of each period
          \item\textbf{n} - Number of payments, and also (in the simple case) the number of periods
          \item\textbf{i} - Compound interest rate \underline{per period} expressed as a decimal {(\%/100)}
          \item\textbf{S} - Accumulated (or Future) Value of the annuity (value at time n)
          \item\textbf{A} - Present Value of the annuity (value at time 0)
\end{description}
\vspace{.1in}

\begin{center}
    \begin{tikzpicture}
        %draw horizontal line with squiggle in middle
        \draw (0,0) -- (5,0);
        \draw[decorate,decoration={snake,pre length=5mm, post length=5mm}] (5,0) -- (7,0);
        \draw (7,0) -- (10,0);

        %draw vertical tick-marks
        \foreach \x in {0,2,4,8,10}
        \draw (\x cm,3pt) -- (\x cm,-3pt);

        %Label "nodes"
        \draw (0,0) node[below=3pt] {$ 0 $} node[above=3pt] {$  $};
        \draw (0,0) node[below=18pt] {$ A $} node[above=3pt] {$  $};

        \draw (2,0) node[below=3pt] {$ 1 $} node[above=3pt] {$ R $};
        \draw (4,0) node[below=3pt] {$ 2 $} node[above=3pt] {$ R $};
        \draw (6,0) node[below=3pt] {$  $} node[above=3pt] {$  $};
        \draw (8,0) node[below=3pt] {$ n - 1 $} node[above=3pt] {$ R $};
        \draw (10,0) node[below=3pt] {$ n $} node[above=3pt] {$ R $};
        \draw (10,0) node[below=18pt] {$ S $} node[above=3pt] {$ R $};
    \end{tikzpicture}
\end{center}
%\vspace{.1in}

\begin{flushleft}
    Then we have
\end{flushleft}
\vspace{-.2in}

\begin{align*}
    S & = R \cdot \frac{(1 + i)^n-1}{i} \\
\end{align*}
\vspace{-.2in}

\begin{flushleft}
    S denotes the ending, or \textbf{accumulated value} of the stream of annuity payments. To value instead
    the payments at the beginning (denoted \textbf{present value}), we just discount S by n periods, as follows:
\end{flushleft}
\vspace{-.1in}

\begin{align*}
    A & = \frac{S}{(1 + i)^n}       \\\\
      & = Sv^n                      \\\\
      & = R \cdot \frac{1 - v^n}{i}
\end{align*}

\begin{flushleft}
    where {\large $ v = \frac{1}{(1+i)} $ }.
\end{flushleft}
\vspace{.1in}

\begin{flushleft}
    \textbf{Note:} \\
\end{flushleft}

\begin{flushleft}
    One reason for distinguishing between principal and interest in loan repayments is that for U.S.
    federal tax purposes, interest is (potentially) tax deductible, whereas principal is not.
\end{flushleft}

\end{document}