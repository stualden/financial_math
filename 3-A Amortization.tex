% LaTeX template for Finance summaries
% Stu Alden
% Amortization
%%%%%%%%%%%%%%%%%%%%%%%%%%%%%%%%%%%%%%%%%%%%%%%%%%%%%%%%%%%%%%%%%%%%%%%%
\documentclass[12pt]{article}

\usepackage[letterpaper,
            left=1in,
            right=1in,
            top=1in,
            bottom=1in,
            footskip=.25in]{geometry}

\usepackage{titling}
\setlength{\droptitle}{-1in}  % Don't need all that space!
\title{\normalfont\ 3-A Amortization} % The article title
\author{} % Don't need author
\date{}  % Suppress date
\pagenumbering{gobble}  % No page numbers required

\usepackage{amsmath}  % For general math purposes

\usepackage{actuarialsymbol}   % for the angle sign on the annuity!

\usepackage{booktabs}   % For tables

\usepackage{tikz}
\usetikzlibrary{decorations.pathmorphing}

\begin{document} %%%%%%%%%%%%%%%%%%%%%%%%%%%%%%%%%%%%%%%%%%%%%%%%%%%%%%

\maketitle % Print the title/author/date block
\vspace{-1.3in}

\begin{flushleft}
    \textbf{Amortization} in its simplest form is the repayment of a debt with level payments ("installments"),
    where each installment covers the full amount of interest that has accrued, plus a portion of the outstanding
    principal.  An \textbf{amortization schedule} is a table with a line for each payment (showing the
    total payment, the interest portion, and the principal portion), and the outstanding principal immediately after
    that payment has been made.  At the beginning, principal is just $ A $ (present value of the annuity), and at
    the end, principal goes to 0.  To illustrate, using \$100 over 4 periods at 10\% per period:
\end{flushleft}

\begin{tabular}{@{ } l l  l  l l l @{ }}    % Six column table
    \toprule
%    {\bfseries Period}&{\bfseries Starting Principal} &  \qquad \qquad & {\bfseries Sides}&{\bfseries Name} \\
    {\bfseries       }&{\bfseries Starting }&{\bfseries Payment}&{\bfseries ---}&{\bfseries ---}&{\bfseries Ending} \\
    {\bfseries Period}&{\bfseries Principal}&{\bfseries Total}&{\bfseries Interest}&{\bfseries Principal}&{\bfseries Principal} \\
%    \cmidrule(lr){1-2}\cmidrule(lr){4-5}
    \cmidrule(lr){1-6}
    1 & \$100.00  & \$31.55 &\$10.00 &\$21.55 &\$78.45 \\
    2 & \$ 78.45  & \$31.55 &\$ 7.85 &\$24.60 &\$54.75 \\
    3 & \$ 54.75  & \$31.55 &\$ 5.47 &\$26.08 &\$28.67 \\
    4 & \$ 28.67  & \$31.55 &\$ 2.87 &\$28.68 &\$ 0.00 \\
    \bottomrule
\end{tabular}

\begin{flushleft}
    If the outstanding principal $ P $ is needed at any point $ k $ during the term of
    the loan, it can be calculated using the standard annuity formulas, either \textit{prospectively} or
    \textit{retrospectively}.  Here, if
\end{flushleft}

\begin{description}
    \item\textbf{n, R, i, A} - Defined previously
    \item\textbf{k    } - Number of periods from beginning, at which to measure outstanding principal (so $ k $ is between 0 and n)
    \item\textbf{P    } - Outstanding principal immediately after the $ k $th payment
    \item\textbf{{$ \ax{ \angl{n-k} i} $}} - Present value of the $ n-k $ remaining payments
    \item\textbf{{$ \sx{ \angl{k} i} $}} - Accumulated Value of $ k $ payments thus far
\end{description}
%\vspace{.1in}

\begin{flushleft}
    Then the prospective formula is
\end{flushleft}
\vspace{-.3in}

\begin{align*}
    P & = R \cdot \ax{ \angl{n-k} i} \\
\end{align*}
\vspace{-.3in}

\begin{flushleft}
    and the retrospective formula is
\end{flushleft}
\vspace{-.3in}

\begin{align*}
    P & = A \cdot (1+i)^k + \sx{ \angl{k} i} \\
\end{align*}
\vspace{-.2in}

\begin{center}
    \begin{tikzpicture}
        %draw horizontal line with squiggle in middle
        \draw (0,0) -- (3,0);
        \draw[decorate,decoration={snake,pre length=5mm, post length=5mm}] (3,0) -- (5,0);
        \draw (5,0) -- (7,0);
        \draw[decorate,decoration={snake,pre length=5mm, post length=5mm}] (7,0) -- (9,0);
        \draw (9,0) -- (10,0);

        %draw vertical tick-marks
        \foreach \x in {0,2,6,8,10}
        \draw (\x cm,3pt) -- (\x cm,-3pt);

        %Label "nodes"
        \draw (0,0) node[below=3pt] {$ 0 $} node[above=3pt] {$  $};
        \draw (0,0) node[below=18pt] {$ A $} node[above=3pt] {$  $};
        \draw (2,0) node[below=3pt] {$ 1 $} node[above=3pt] {$ R $};
        \draw (6,0) node[below=3pt] {$ k $} node[above=3pt] {$ R $};
        \draw (6,0) node[below=18pt] {$ P $} node[above=3pt] {$  $};
        \draw (10,0) node[below=3pt] {$ n $} node[above=3pt] {$ R $};
        \draw (10,0) node[below=18pt] {$ 0 $} node[above=3pt] {$ R $};
    \end{tikzpicture}
\end{center}

\begin{flushleft}
    \textbf{Notes:} \\
\end{flushleft}

\begin{itemize}
    \item One reason for distinguishing between principal and interest in loan repayments is that for U.S.
          federal tax purposes, interest is (potentially) tax deductible, whereas principal is not.
    \item Knowing outstanding principal is also useful--for example, the market value of a home minus the
          outstanding principal on a mortgage represents the homeowner's \textbf{equity} in the home.
\end{itemize}

\end{document}