% LaTeX template for Finance summaries
% Stu Alden
%----------------------------------------------------------------------------------------
\documentclass[12pt]{article}

\usepackage{amsmath}

\usepackage{titling}
\setlength{\droptitle}{-1.5in}  % Don't need all that space!

\usepackage{tikz}
\usetikzlibrary{decorations.pathmorphing}


\title{\normalfont\ 1-A-1 Simple Interest} % The article title
\author{} % Don't need author
\date{}  % Suppress date
\pagenumbering{gobble}  % No page numbers required

\begin{document}

\maketitle % Print the title/author/date block

\vspace{-1in}

\begin{flushleft}
    \textbf{Simple Interest} is just linear growth in the amount of money.
    The simple interest rate is the rate of growth over each period.
\end{flushleft}

\begin{description}
    \item\textbf{A} - amount at the beginning, or principal
    \item\textbf{S} - amount at the end, or final amount
    \item\textbf{i} - interest rate per period expressed as a decimal {(\%/100)}
    \item\textbf{n} - number of periods (could be fractional)
\end{description}

\begin{flushleft}
    Then we have
\end{flushleft}

\vspace{-.5in}

\begin{align*}
    Amount \: of \: Interest & = I                 \\
                             & = A \cdot i \cdot n
\end{align*}

\begin{flushleft}
    and
\end{flushleft}
\vspace{-.5in}

\begin{align*}
    S & = A + I     \\
      & = A(1 + in)
\end{align*}

\begin{flushleft}
    or
\end{flushleft}

\begin{align*}
    A & = \frac{S}{(1+in)}
    %  & = A(1 + in)
\end{align*}
\vspace{.25in}


\begin{center}
    \begin{tikzpicture}
        %draw horizontal line with squiggle in middle
        \draw (0,0) -- (5,0);
        \draw[decorate,decoration={snake,pre length=5mm, post length=5mm}] (5,0) -- (7,0);
        \draw (7,0) -- (10,0);

        %draw vertical lines
        \foreach \x in {0,2,4,8,10}
        \draw (\x cm,3pt) -- (\x cm,-3pt);

        %draw nodes
        \draw (0,0) node[below=3pt] {$ 0 $} node[above=3pt] {$ A $};
        \draw (2,0) node[below=3pt] {$ 1 $} node[above=3pt] {$ A(1+i) $};
        \draw (4,0) node[below=3pt] {$ 2 $} node[above=3pt] {$ A(1+2i) $};
        % \draw (6,0) node[below=3pt] {$  $} node[above=3pt] {$  $};
        \draw (8,0) node[below=3pt] {$ n - 1 $} node[above=3pt] {$ A(1+(n-1)i) $};
        \draw (10,0) node[below=3pt] {$ n $} node[above=3pt] {$ S $};

        \newline
        \newline

    \end{tikzpicture}
\end{center}

\begin{center}
    {Read this way} {$\Rightarrow$}
\end{center}

%\vspace{.125in} %5mm vertical space

\begin{flushleft}
    \textbf{Notes:} \\
\end{flushleft}


\begin{itemize}
    \item i is a percentage of the \underline{beginning} amount, but it is credited at the \underline{end} of each period.
    \item Interest rates are annual and the period is a year unless otherwise stated
\end{itemize}

\end{document}
