% LaTeX template for Finance summaries
% Stu Alden
%----------------------------------------------------------------------------------------
% Compound Interest Continued
\documentclass[12pt]{article}

\usepackage[letterpaper,
            left=1in,
            right=1in,
            top=1in,
            bottom=1in,
            footskip=.25in]{geometry}

\usepackage{titling}
\setlength{\droptitle}{-1in}  % Don't need all that space!
\title{\normalfont\ 1-B-2 More on Compound Interest} % The article title
\author{} % Don't need author
\date{}  % Suppress date
\pagenumbering{gobble}  % No page numbers required

\usepackage{amsmath}

\usepackage{tikz}
\usetikzlibrary{decorations.pathmorphing}

\begin{document}

\maketitle % Print the title/author/date block

\vspace{-1in}

\begin{flushleft}
    \textbf{Cashflow Equestions}
\end{flushleft}
\vspace{.1in}

\begin{itemize}
    \item Different cashflows can be compared and equated, at a particular compound interest rate, by
          accumulating and/or discounting all flows to a single ``measurement'' or ``comparison'' date.
    \item Unlike simple interest and simple discount, compound interest's \underline{geometric} growth
          pattern provides the same results regardless of your choice of comparison date.
\end{itemize}
\vspace{1.5in}

\begin{flushleft}
    \textbf{Equivalent Rates}
\end{flushleft}
\vspace{.1in}


\begin{itemize}
    \item For any choice of interest rate and compounding period, an equivalent rate can be determined
          with a different rate and a different compounding period.
    \item Unfortunately, a wide variety of terminology is used, which can cause confusion.
    \item To keep th ings straight, the first step is always to identify the \underline{period} and
          the precise \underline{rate per period}
\end{itemize}
\vspace{.1in}

% \begin{flushleft}
%     Then we have
% \end{flushleft}
% \vspace{-.1in}

% \begin{align*}
%     S & = A(1 + i)^n
% \end{align*}
% \vspace{-.1in}

% \begin{flushleft}
%     and
% \end{flushleft}
% \vspace{-.1in}

% \begin{align*}
%     A & = \frac{S}{(1 + i)^n} \\
%       & = S(1+i)^{-1}         \\
%       & = Sv^n
% \end{align*}

% \begin{flushleft}
%     where
% \end{flushleft}
% \vspace{-.1in}

% \begin{align*}
%     v & = \frac{1}{1 + i} = (1+i)^{-1}
% \end{align*}
% \vspace{.25in}

% \begin{center}
%     \begin{tikzpicture}
%         %draw horizontal line with squiggle in middle
%         \draw (0,0) -- (5,0);
%         \draw[decorate,decoration={snake,pre length=5mm, post length=5mm}] (5,0) -- (7,0);
%         \draw (7,0) -- (10,0);

%         %draw vertical tick-marks
%         \foreach \x in {0,2,4,8,10}
%         \draw (\x cm,3pt) -- (\x cm,-3pt);

%         %Label "nodes"
%         \draw (0,0) node[below=3pt] {$ 0 $} node[above=3pt] {$ A $};
%         \draw (2,0) node[below=3pt] {$ 1 $} node[above=3pt] {$ A(1+i) $};
%         \draw (4,0) node[below=3pt] {$ 2 $} node[above=3pt] {$ A(1+i)^2 $};
%         \draw (6,0) node[below=3pt] {$  $} node[above=3pt] {$  $};
%         \draw (8,0) node[below=3pt] {$ n - 1 $} node[above=3pt] {$ A(1+i)^{n-1} $};
%         \draw (10,0) node[below=3pt] {$ n $} node[above=3pt] {$ S $};
%     \end{tikzpicture}
% \end{center}
% \vspace{.1in}

% \begin{flushleft}
%     \textbf{Notes:} \\
% \end{flushleft}


% \begin{itemize}
%     \item Multiplying by $ (1+i)^t $ (which is $>$ 1) is called \textbf{``accumulating t periods.''}
%     \item Mutiplying by $ v^t $ (which is $<$ 1) is called \textbf{``discounting t periods.''}
%     \item Knowing any three of the variables $A$, $S$, $n$ and $i$, you can find the fourth.
% \end{itemize}

\end{document}
