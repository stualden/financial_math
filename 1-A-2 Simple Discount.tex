% LaTeX template for Finance summaries
% Stu Alden
%----------------------------------------------------------------------------------------
\documentclass[12pt]{article}

\usepackage{amsmath}

\usepackage{titling}
\setlength{\droptitle}{-1.5in}

\usepackage{tikz}
\usetikzlibrary{decorations.pathmorphing}

\title{\normalfont\ 1-A-2 Simple Discount} % The article title
\author{}  % Could include but only as small footnote
\date{}  % Suppress date
\pagenumbering{gobble}

\begin{document}

\maketitle % Print the title/author/date block

\vspace{-1in}

\begin{flushleft}
    \textbf{Simple Discount} is also linear growth, just like Simple Interest.
    They are equivalent concepts, but the terminology and labeling are different.
\end{flushleft}

\begin{description}
    \item\textbf{A} - amount at the beginning, or "principal" or "proceeds"
    \item\textbf{S} - amount at the end, or "final amount"
    \item\textbf{d} - discount rate \underline{per period}, expressed as a decimal {(\%/100)}
    \item\textbf{n} - number of periods
\end{description}

\begin{flushleft}
    Then we have
\end{flushleft}

\vspace{-.25in}

\begin{align*}
    Amount \: of \: Interest & = I                 \\
                             & = S \cdot d \cdot n
\end{align*}

\begin{flushleft}
    and
\end{flushleft}

\vspace{-.25in}

\begin{align*}
    A & = S - I     \\
      & = S - Sdn)  \\
      & = S(1 - dn)
\end{align*}

\vspace{.25in}

\begin{tikzpicture}
    %draw horizontal line with squiggle in middle
    \draw (0,0) -- (3,0);
    \draw[decorate,decoration={snake,pre length=5mm, post length=5mm}] (3,0) -- (5,0);
    \draw (5,0) -- (10,0);

    %draw vertical lines
    \foreach \x in {0,2,6,8,10}
    \draw (\x cm,3pt) -- (\x cm,-3pt);

    %    {\tiny Text of first line}
    %draw nodes
    \draw (0,0) node[below=3pt] {$ 0 $} node[above=3pt] {$ A $};
    \draw (2,0) node[below=3pt] {$ 1 $} node[above=3pt] {$ S-(n-1)dS $};
    \draw (6,0) node[below=3pt] {$ n-2 $} node[above=3pt] {$ S-2dS $};
    \draw (8,0) node[below=3pt] {$ n-1 $} node[above=3pt] {$ S-dS $};
    \draw (10,0) node[below=3pt] {$ n $} node[above=3pt] {$ S $};

    \newline
    \newline

\end{tikzpicture}

\begin{center}
    {$\Leftarrow$}  {Read this way}
\end{center}

\vspace{.25in}

\begin{flushleft}
    \textbf{Notes:} \\
\end{flushleft}


\begin{itemize}
    \item d is a percentage of the \underline{final} amount S, whereas i (simple interest) is a percentage of the \underline{beginning} amount A
    \item We can think of this as working backward and \underline{subtracting} dS each period.  With simple interest, we work forward and \underline{add} Ai each period
    \item Like interest rates, discount rates are annual, and the period is a year, unless otherwise stated
\end{itemize}

\end{document}
